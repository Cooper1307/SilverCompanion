\documentclass[UTF8,zihao=-4,scheme=chinese]{ctexart}
\usepackage{geometry}
\usepackage{titlesec}
\usepackage{setspace}
\usepackage{booktabs}
\usepackage{graphicx}
\usepackage{array}
\usepackage{float}
\usepackage{hyperref}

% 页面设置: A4纸, 上2.5cm, 下2.5cm, 左3.1cm, 右3.1cm
\geometry{a4paper, top=2.5cm, bottom=2.5cm, left=3.1cm, right=3.1cm}

% 行间距: 1.3倍
\setstretch{1.3}

% 标题格式设置
% 大标题: 三号, 加粗. ctex 下使用 \heiti
\newcommand{\mainboundtitle}{\bfseries\heiti\zihao{3}}

% 一级标题: 仿宋(或默认), 小三. ctex 下建议使用 \sifang
\titleformat{\section}{\zihao{-3}\bfseries}{\thesection}{1em}{}
% 二级标题: 小四, 加粗
\titleformat{\subsection}{\zihao{4}\bfseries}{\thesubsection}{1em}{}

\begin{document}

\begin{center}
    {\mainboundtitle 上海杉达学院人工智能通识教育课程项目实践报告}
\end{center}

\vspace{-2cm}

\noindent \textbf{课程名称:} 人工智能应用与实践 \\
\noindent \textbf{项目主题:} AI助力老龄化社区治理 \\
\noindent \textbf{指导教师:} 

\section{项目基本信息}

\subsection{项目基础信息}
\begin{table}[H]
\centering
\begin{tabular}{|m{4cm}|m{8cm}|}
\hline
\bfseries 项目信息 & \bfseries 内容 \\ \hline
项目名称 & 智守银夕 - 基于“15分钟生活圈”的社区适老化智慧助理 \\ \hline
智能体名 & 银龄小助手 (SilverCompanion) \\ \hline
项目负责人 & 陈露 \\ \hline
\end{tabular}
\end{table}

\subsection{团队成员信息}
\begin{table}[H]
\centering
\zihao{5}
\begin{tabular}{|c|c|l|c|p{7cm}|}
\hline
序号 & 姓名 & 学号 & 专业 & 团队分工 \\ \hline
1 & 陈露 & f23061234 & 英语 & \bfseries 组长:项目总负责人,负责整体架构设计、进度把控及“医工交叉”技术路线制定。 \\ \hline
2 & 袁帅 & f22061436 & 英语 & \bfseries 文档与交互:负责需求梳理、文档撰写及基于上海地方法规的Prompt设计。 \\ \hline
3 & 孙嘉宁 & f22076115 & 工程 & \bfseries 技术开发:负责智能体后端逻辑配置及API接口调试。 \\ \hline
4 & 黄嘉琳 & f23111107 & 医技 & \bfseries 专业顾问与数据:负责康复护理知识库的收集、清洗与专业性校验。 \\ \hline
5 & 高子伟 & f22023221 & 电商 & \bfseries 前端展示:负责“适老化”界面设计与排版。 \\ \hline
6 & 薛尉廷 & f22076223 & 工程 & \bfseries 测试优化:负责多轮对话测试及Bad Case分析修复。 \\ \hline
7 & 时铭阳 & f22076217 & 工程 & \bfseries 部署运维:负责发布上线及环境配置。 \\ \hline
\end{tabular}
\end{table}

\section{项目概述}

\subsection{问题背景}
截止 2025 年底,上海户籍 60 岁及以上老年人口预计将达到 600 万人,“纯老家庭”和独居老人数量呈指数级增长。尽管上海市大力推行“15 分钟社区生活圈”,但在社区服务的“最后一米”,仍存在严重的人力缺口。
解决“健康咨询难、情感陪伴缺、紧急响应慢”的三大痛点,是本项目关注的核心问题。

\subsection{解决方案}
本项目依托我校国际医学技术学院与信息科学与技术学院的“医工融合”学科背景,基于大语言模型技术,开发了“银龄小助手”智能体。核心功能包括:极简语音交互、社区政策通(对接长护险政策)、健康守护哨兵(含紧急词汇识别预警)及情感陪伴。

\subsection{服务对象}
核心用户为上海市“15分钟生活圈”内的居家养老群体;辅助用户为街道网格员、社区医生、老人子女。

\subsection{项目价值}
本项目不仅为社区提供了24小时在线的“数字社工”,填补了服务空白,还通过技术手段降低了养老服务成本,为数字化社区养老服务提供了可复制的基层解决方案。

\section{需求与挑战分析}

\begin{table}[H]
\centering
\zihao{-4}
\begin{tabular}{|p{3cm}|p{5cm}|p{5cm}|}
\hline
 需求类型 &  具体内容描述 &  关键挑战与考量 \\ \hline
业务目标 & 打造“懂医疗、懂政策、懂老人”的高质量智能体。 & 重点在于如何确保AI对上海各区差异化养老政策回答准确? \\ \hline
用户需求 & 大字版界面与极简语音操作,降低数字门槛。 & 如何解决老人对“AI生成内容”的天然不信任感? \\ \hline
数据需求 & 输入:健康问答、政策文件;输出:有温度的建议。 & 如何收集敏感公共数据同时保护用户隐私? \\ \hline
\end{tabular}
\end{table}

\section{技术方案设计}

\subsection{数据准备}
数据集来源于通用医疗指南、上海市官方养老政策汇编及我校康复专业提供的特色知识库。处理流程包括数据清洗、向量化(Embedding)并存入本地知识库。

\subsection{模型与平台选择}
选择基于 \textbf{Aliyun Bailian} 与 \textbf{OpenRouter} 的分布式多路由架构。核心模型为 \textbf{Qwen-Max} (国产旗舰) 与 \textbf{Llama 3.3 70B}。理由是实现了自动故障切换机制,确保服务永不掉线。

\subsection{业务与工作流设计}
用户发起语音请求后,ASR系统转写为文字,智能中控进行意图识别,根据策略检索相应的健康或政策知识库,最后由大模型整合生成回复。

\section{开发过程与关键决策}

\subsection{开发拆解}
开发分为场景定义、知识库构建、后端API开发与前端适老化适配。技术难点在于解决跨域冲突与实现多模型容灾备份。

\subsection{Prompt设计示例}
采用“角色沉浸 + 情绪链”思路。核心指令要求模型扮演“金牌社工小张”,使用上海晚辈口吻,并设定“一句话不超过15个字”的精简标准。

\subsection{关键决策记录}
技术难题上,通过增加元数据年份权重解决了 RAG 检索政策过期的问题。伦理决策上,坚持不直接调用120接口,而是通过语音强引导,保留“人”作为最终决策者。

\section{项目成效与部署}
项目已发布内测链接。经过测试,政策回答准确率达 92\%,紧急响应拦截率 100\%。界面功能如下:
\begin{itemize}
    \item \textbf{适老化首页}:极简大按钮设计,集成 12.31 实时天气系统。
    \item \textbf{政策解读与语音交互}:展示针对“静安区助餐补贴”的 2025 最新政策答复,包含语音录制组件。
    \item \textbf{多模态健康建议}:针对慢性病老人的个性化饮食建议卡片,通过图标提升信息传达效率。
\end{itemize}


\section{项目反思与迁移}
实践证明,“信任”比“功能”更重要。未来可将本项目的语义理解与多模态能力迁移至“AI助力视障人士智慧出行”场景。

\section{伦理合规性自查}
项目已完成数据隐私保护、算法公平性、用户知情同意及结果可解释性自查。严格限制品牌诱导消费,确保适老化设计合规。

\section{项目团队分工}
团队成员在技术掌握、团队协作、创新思维、社会责任及伦理意识五个维度均表现优异。

\section{参考资料}
1. 上海市智慧养老三年行动方案;\\
2. 《15分钟社区生活圈规划导则》;\\
3. 《中国居民膳食指南(2022)》。

\end{document}
